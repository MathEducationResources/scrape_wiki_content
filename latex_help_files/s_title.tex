\thispagestyle{empty}

\vspace{-1.7cm}

\begin{center}
{\huge \bf Full Solutions \\[0.3cm] \course \exam}
\end{center}
\begin{flushright}\today\end{flushright}

\vspace{-0.8cm}

\subsection{How to use this resource}

\begin{itemize}
\item When you feel reasonably confident, simulate a full exam and grade your solutions. This document provides full solutions that you can use to grade your work.
\item If you're not quite ready to simulate a full exam, we suggest you thoroughly and slowly work through each problem. To check if your answer is correct, without spoiling the full solution, we provide a pdf with the final answers only. \href{http://matheducationresources.github.io/pdf_version/}{Download the document with the final answers here.}
\item Should you need more help, check out the hints and video lecture on the \href{http://www.math-education-resources.com}{Math Education Resources}.
\end{itemize}


\section{Tips for Using Previous Exams to Study: Exam Simulation}

\emph{Resist the temptation to read any of the solutions below before completing each question by yourself first! We recommend you follow the guide below.}

\begin{enumerate}
\item {\bf Exam Simulation:} When you've studied enough that you feel reasonably confident, \href{\examURL}{print the raw exam (click here)} without looking at any of the questions right away.  Find a quiet space, such as the library, and set a timer for the real length of the exam (usually 2.5 hours).  Write the exam as though it is the real deal.
\item {\bf Reflect on your writing:} Generally, reflect on how you wrote the exam.  For example, if you were to write it again, what would you do differently?  What would you do the same?  In what order did you write your solutions? What did you do when you got stuck?
\item {\bf Grade your exam:} Use the solutions in this pdf to grade your exam.  Use the point values as shown in the original exam.
\item {\bf Reflect on your solutions:} Now that you have graded the exam, reflect again on your solutions.  How did your solutions compare with our solutions?  What can you learn from your mistakes?
\item {\bf Plan further studying:} Use your mock exam grades to help determine which content areas to focus on and plan your study time accordingly. Brush up on the topics that need work:
\begin{itemize}
\item Re-do related homework and webwork questions.
\item The Math Education Resources offers mini video lectures on each topic.
\item Work through more previous exam questions thoroughly without using anything that you couldn't use in the real exam. Try to work on each problem until your answer agrees with our final result.
\item Do as many exam simulations as possible.
\end{itemize}
Whenever you feel confident enough with a particular topic, move on to topics that need more work. Focus on questions that you find challenging, not on those that are easy for you. Always try to complete each question by yourself first.
\end{enumerate}

\vfill

{\small This pdf was created for your convenience when you study Math and prepare for your final exams. All the content here, and much more, is freely available on the \href{http://www.math-education-resources.com}{Math Education Resources}.}

\vfill

\begin{multicols}{2}
\hfill \begin{minipage}{0.45\textwidth}This is a free resource put together by the \href{http://www.math-education-resources.com}{Math Education Resources}, a group of volunteers with a desire to improve higher education. You may use this material under the \href{https://creativecommons.org/licenses/by-nc-sa/4.0/}{Creative Commons Attribution-NonCommercial-ShareAlike 4.0 International licence}.

\end{minipage}

\columnbreak

\begin{center}
\includegraphics[width=0.3\textwidth]{MER_penguin_left.png}
\end{center}

\vfill

\end{multicols}
