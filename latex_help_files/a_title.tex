\thispagestyle{empty}
\forkme[east]

\vspace{-1cm}

\begin{center}
{\huge \bf Final Answers \\[0.3cm] \course \exam}
\end{center}
\begin{flushright}\today\end{flushright}

\vspace{-0.5cm}

\subsection{How to use this resource}

\begin{itemize}
\item When you feel reasonably confident, simulate a full exam and grade your solutions. \href{http://matheducationresources.github.io/pdf_version/}{For your grading you can get the full solutions here.}
\item If you're not quite ready to simulate a full exam, we suggest you thoroughly and slowly work through each problem. Use this document with the final answers only to check if your answer is correct, without spoiling the full solution.
\item Should you need more help, check out the hints and video lecture on the \href{http://www.math-education-resources.com}{Math Education Resources}.
\end{itemize}


\section{Tips for Using Previous Exams to Study: Work through problems}

\emph{Resist the temptation to read any of the final answers below before completing each question by yourself first! We recommend you follow the guide below.}


\begin{enumerate}
\item {\bf How to use the final answer:} \emph{The final answer is not a substitution for the full solution!} The final answer alone will not give you full marks. The final answer is provided so that you can check the correctness of your work without spoiling the full solution.
\begin{itemize}
\item To answer each question, only use what you could also use in the exam. \href{\examURL}{Download the raw exam here}.
\item If you found an answer, how could you verify that it is correct from your work only? E.g. check if the units make sense, etc.  Only then compare with our result.
\item If your answer is correct: good job! Move on to the next question.
\item Otherwise, go back to your work and check it for improvements. Is there another approach you could try? If you still can't get to the right answer, you can check the full solution on the \href{http://www.math-education-resources.com}{Math Education Resources}.
\end{itemize}

\item {\bf Reflect on your work:} Generally, reflect on how you solved the problem. Don't just focus on the final answer, but whether your mental process was correct. If you were stuck at any point, what helped you to go forward? What made you confident that your answer was correct? What can you take away from this so that, next time, you can complete a similar question without any help?

\item {\bf Plan further studying:} Once you feel confident enough with a particular topic, move on to topics that need more work. Focus on questions that you find challenging, not on those that are easy for you. Once you are ready to tackle a full exam, follow the advice for the \href{http://matheducationresources.github.io/pdf_version/}{full exam (click here)}.
\end{enumerate}

{\bf Please note that all final answers were extracted automatically from the full solution. It is possible that the final answer shown here is not complete, or it may be missing entirely. In such a case, please notify \href{mailto:mer-wiki@math.ubc.ca}{\nolinkurl{mer-wiki@math.ubc.ca}}. Your feedback helps us improve.}

{\small This pdf was created for your convenience when you study Math and prepare for your final exams. All the content here, and much more, is freely available on the \href{http://www.math-education-resources.com}{Math Education Resources}.}

\vfill

\begin{multicols}{2}
\hfill \begin{minipage}{0.45\textwidth}This is a free resource put together by the \href{http://www.math-education-resources.com}{Math Education Resources}, a group of volunteers that turn their desire to improve education into practice. You may use this material under the \href{https://creativecommons.org/licenses/by-nc-sa/4.0/}{Creative Commons Attribution-NonCommercial-ShareAlike 4.0 International licence}.

\end{minipage}

\columnbreak

\begin{center}
\includegraphics[width=.14\textwidth]{MERS_penguin_left_min.png}
\end{center}

\end{multicols}

\vfill
